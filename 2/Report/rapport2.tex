\documentclass[a4paper,English,10pt]{article}
\usepackage[utf8]{inputenc}
\usepackage[T1]{fontenc}
%% \usepackage[English]{babel}     
\usepackage{graphicx,mathpple, textcomp, varioref}
\usepackage{fullpage}
\usepackage{fancyhdr}
\usepackage{lastpage}
\usepackage{hyperref}
\usepackage{amsmath}
\usepackage{braket}
\usepackage{enumitem}
\usepackage{booktabs}

\title{Fys4110: Project 2}
\author{Knut Halvor Helland}
\pagestyle{fancyplain}
\fancyhf{}
\renewcommand{\headrulewidth}{0pt}
\fancyfoot[R]{\thepage/\pageref{LastPage}}
\tolerance = 5000
\hbadness = \tolerance
\pretolerance = 2000
\setlength{\headheight}{20pt}

\newcommand{\unit}[1]{\; \mathrm{#1}}
\newcommand{\bb}[1]{\boldsymbol{#1}}
\newcommand{\p}{\partial}
\newcommand{\dd}{\mathrm{d}}
\newcommand{\ddt}[2]{\frac{\dd #1}{\dd #2}}
\newcommand{\dndt}[3]{\frac{\dd^{#3} #1}{\dd #2^{#3}}}
\newcommand{\pddt}[2]{\frac{\p #1}{\p #2}}
\newcommand{\pndt}[3]{\frac{\p^{#3} #1}{\p #2^{#3}}}
\newcommand{\Rar}{\Rightarrow}
\newcommand{\rar}{\rightarrow}
\newcommand{\uar}{\uparrow}
\newcommand{\lar}{\leftarrow}
\newcommand{\dar}{\downarrow}
\newcommand{\lagr}{\mathscr{L}}
\newcommand{\ham}{\mathcal{H}}
\newcommand{\id}{\bb{1}}
\newcommand{\deldt}[2]{\frac{\delta #1}{\delta #2}}
\newcommand{\be}{\begin{equation}}
\newcommand{\ee}{\end{equation}}
\newcommand{\f}{\frac}
\newcommand{\Det}{\mathrm{Det}}
\newcommand{\sgn}[1]{(-1)^{|#1|}}


\renewcommand{\bar}{\overline}
\renewcommand{\braket}{\Braket}

\begin{document}
\maketitle{}
\begin{abstract}
$\ldots$
\end{abstract}


\section{Introduction}
In this project I continue the study of many interacting particles in an isotropic two dimensional harmonic oscillator.
In project 1 \cite{proj1} I used Hartree-Fock methods to construct the Slater determinant of linear combinations of single
particle non-interacting states that minimized the energy. In this project I would like to improve this estimate of the ground state
by adding a Jastrow factor. Then the energy cannot be found with Hartree-Fock methods, but rather from direct integration of the Hamiltonean.
I will do this using the Metropolis algorithm to pick out integration points.
I will add a Jastrow factor with a free parameter, minimize the energy with respect to this parameter, and
then calculate the energy of this new approximation of the ground state.
In addition I will study the 2 particle case with a singly parametrized symmetric wavefunction instead of the Slater determinant.


\section{Physical Problem}
We will study several Coulomb-interacting electrons in an isotropic two dimensional harmonic oscillator.
The full Hamiltonean of the problem with $N$ particles and using atomic units is
\be
H = \sum_i^N -\f{1}{2}\nabla^2_i + \f{1}{2}\omega^2r^2_i + \sum_{i<j}^N\f{1}{r_{ij}}, \label{ham}
\ee
where $r_{ij} = |\bb{r}_i-\bb{r}_j|$.
The non-interacting hamiltonean
\be
H_0 = \sum_i^N -\f{1}{2}\nabla^2_i + \f{1}{2}\omega^2r^2_i,\label{ham0}
\ee
has as an analytical ground state solution (up to normalisation) the slater determinant
\be
\mathcal{A}\left(\psi_1\ldots\psi_N;\bb{r}_1\ldots\bb{r}_N\right) =
\left|\begin{matrix}
  \psi_1(\bb{r}_1)&\cdots&\psi_N(\bb{r}_1)\\
  \vdots&\ddots&\vdots\\
  \psi_1(\bb{r}_N)&\cdots&\psi_N(\bb{r}_N)\\
  \end{matrix}\right|,
\ee
where
\be
\psi_i(\bb{r}) = \chi_i\psi_{n_x,n_y}(\bb{r}) = \chi_iH_{n_x}(\sqrt{\omega}x)H_{n_y}(\sqrt{\omega}y)\exp(-\f{r^2}{2}),\label{spwf}
\ee
are the $N$ lowest energy single particle wave eigenstates.
The general ground state for the interacting case is not known.


\section{Trial wave functions}
\subsection{2 Particle Case}
For the two particle case I will aproximate the ground state with the parametrized trial wavefunction
\be
\psi_T = \exp\left(-\alpha\omega(r_1^2 + r_2^2) -\f{r_{12}}{1+\beta r_{12}}\right),\label{2pw}
\ee
with $\alpha$ and $\beta$ as free parameters.

\subsection{Slater case}

For the many particle case we will use the non-interacting ground state Slater determinant for a modified potential of strength $\alpha\omega$
multiplied with a Jastrow factor with one free parameter.
We could used Slater determinants from linear combinations of different non-interacting single particle states as found in \cite{proj1} by Hartree-Fock methods,
however this turned out to be too complicated too do in our timeframe. Instead we will compare the Hartree-Fock approach to adding a Jastrow factor to this modified ground state.

\be
\psi_T =  \mathcal{A}\left(\psi_1(\alpha)\ldots\psi_N(\alpha);\bb{r}_1\ldots\bb{r}_N\right)\prod_{i<j}^N\exp\left(-\f{a_{ij}r_{ij}}{1+\beta r_{ij}}\right),
\ee
where  $\psi_i$ is the non-interacting single particle state with the $i$th lowest energy,
and $a_{ij}$ is $1$ when the particles have anti-paralell spins and $1/3$ when paralell.
Since we are talking about spin $1/2$ electrons there are two spin configurations for each spatial configuration for the single particle states that go into the Slater determinant.
However the hamiltonean \ref{ham} is independent of spin. Wen can exploit this to simplify the wavefunction we use. If we let
the $N/2$ first particles be in one spin state and the next $N/2$ in the other we may use a product of Slater determinants with only the particles in the same spin state in each:
\be
\psi_T = \mathcal{A}\left(\psi_{1,\dar}(\alpha)\ldots\psi_{N/2,\dar}(\alpha);\bb{r}_1\ldots\bb{r}_{N/2}\right)\mathcal{A}\left(\psi_{1\uar}(\alpha)\ldots\psi_{N/2\uar}(\alpha);\bb{r}_{N/2+1}\ldots\bb{r}_N\right)\prod_{i<j}^N\exp\left(-\f{a_{ij}r_{ij}}{1+\beta r_{ij}}\right).\label{npw}
\ee
In the 2 particle case this reduces to equation \ref{2pw}.


\section{Monte-Carlo Integration}

% wavefunction, slater, jastrow
In this project I will estimate the energy of the trial wavefunctions with Monte-Carlo (MC) integration.
MC integration is a method of estimating the value of integrals on the form
\be
\braket{F} = \int F(x_1,\ldots,x_N)p(x_1,\ldots,x_N)\prod_{k=1}^N\dd x_k, \label{MCI}
\ee
where \(p(x_1,\ldots,x_N)\) is a probability density function.
MC integration is then based on drawing points $\bb{r}_i$ according to $p$ and evaluating $F$ in these points
\be
F_i \equiv F(\bb{r}_i).
\ee
Then defining
\be
\bar{F} \equiv \f{1}{N}\sum_{i=1}^N F_i
\ee
we have that
\be
\braket{F} = \lim_{N\rar\infty}\bar{F}.
\ee
Since we cannot draw infinite points we use $\bar{F}$ as an estimate of $\braket{F}$.

In order to estimate an integral in this way we thus need a method to draw positions according to $p$,
a way to compute $F_i$ and a way to estimate the error in the integration.





\subsection{The Metropolis Algorithm}

%% The Metropolis algorithm is an algorithm for computing expectation values for functions of many variables efficiently.
%% It is based on semi-randomly walking through the integration space and finding the value of the function at each point.
%% Let \(F(x_1,\ldots,x_N)\) be the function we wish to find the expectation value of. Then
%% \be
%% \braket{F} = \f{\int F(x_1,\ldots,x_N)p(x_1,\ldots,x_N)\prod_{k=1}^N\dd x_k}{\int p(x_1,\ldots,x_N)\prod_{k=1}^N\dd x_k}, \label{exp}
%% \ee
%% where \(p(x_1,\ldots,x_N)\) is the (generally) non-normalized probability density function. In our case \(p\) is \(|\psi|^2\).
%% One advantage with this method is that it does not require explicit normalization of the probalility density.
The Metropolis algorithm is a way to draw points according to a probability ensity function $p$. It is based on semi randomly-walking through
the space. One advantage of the method is that the pdf does not need to be normalised, which is convenient.
We can thus rewrite equation \ref{MCI} to
\be
\braket{F} = \f{\int F(x_1,\ldots,x_N)p(x_1,\ldots,x_N)\prod_{k=1}^N\dd x_k}{\int p(x_1,\ldots,x_N)\prod_{k=1}^N\dd x_k}. \label{exp}
\ee
One disadvantage is that
as the each point is only a relatively small step from the last the correlations between points are high enough that we have to take them into account.


\subsubsection{Detailed Balance}
The Metropolis algorithm may be derived by demanding that the Markov chain exhibits detailed balance.
The criterion for detailed balance is
\be
P(a)P(b|a) = P(b)P(a|b),
\ee
or rewritten
\be
\f{P(a)}{P(b)} = \f{P(a|b)}{P(b|a)}.
\ee
We may split \(P(a|b) = G(a|b)A(a|b)\), where \(G(a|b)\) is the probability of proposing a move from $b$ to $a$, while
$A(a|b)$ is the probability of accepting a proposed move from $b$ to $a$. The the detailed balance requirement may be rewritten as
\be
\f{A(a|b)}{A(b|a)} = \f{P(a)G(b|a)}{P(b)G(a|b)}.
\ee
Now we choose an acceptance ratio that satisfies this requirement. The Metropolis choice is to use
\be
A(a|b) = \mathrm{min}\left(1,\f{P(a)G(b|a)}{P(b)G(a|b)}\right).\label{arat}
\ee

So the Metropolis algorithm for drawing points from a probability distrobution is
\begin{enumerate}
\item
  Draw a proposed move from the proposal distribution.
\item
  Evaluate the acceptance ratio $a$ from equation \ref{arat}.
\item
  Draw a random number $0 \leq r < 1$ from a uniform distribution.
\item
  If $a>r$ accept the move. Else reject the move.
\item
  Save position
\item
  return to point 1.
\end{enumerate}
  
  

\subsubsection{Symmetric Proposal Density}

If the proposal probality $G(a|b) = G(b|a)$ the proposal distribution is called symmetric and drops out of the acceptance ratio.
Thus there is less to calculate for each loop in the algortihm. However symmetric proposal densities lead to many proposed steps being rejected,
and thus to higher corelations between the points.


\subsubsection{Importance Sampling}

If we instead choose a non-symmetric proposal distribution we may choose one to maximise the acceptancerate.
If we choose a proposal distribution so that the probability of proposing a move into an area with a higher probability is higher than
the probability of proposing a move into an area with a lower probability, in other words
if $P(a)>P(b)$ then $G(a|b)>G(b|a)$, we will increase the acceptance rate. We need to make sure that the proposal distribution preserves ergodicity,
which means that the whole space may be reached from any point with enought steps. So the probability of proposing a move into an area with lower probability must be non-zero.
The obvious way to ensure proposals into higher probability is to use the gradient of $p$:
\be
\f{1}{p}\nabla p = \f{1}{|\psi_T|^2}\nabla|\psi|^2 = \f{2}{\psi_T}\nabla\psi \equiv 2F, \label{eq:qforce} 
\ee
where I have used that the wavefunction is real. If we only move one particle at a time in one direction at a time we only the derivative with respect to
that particle in that dimension. 
One choice that uses this and preserves ergodicity is
\be
\bb{x}_{i,n+1} = \bb{x}_{i,n} + \sigma \chi + \sigma^2F(x_{i,n}),\label{eq:prop}
\ee
where $\chi$ is a random variable from a gaussian distribution about $0$ with standard deviation $1$.
From \cite{mortenbok} the proposal distribution from this rule is
\be
G(x_{i,n+1}|x_{i,n}) = \exp\left(-\f{(x_{i,n+1} - x_{i,n} - \sigma^2F(x_{i,n}))^2}{2\sigma^2}\right).
\ee
\subsection{Local Energy}
In quantum mechanics the probability distribution is given by $|\psi|^2$ and the expectation value for an operator $O$ is given by
\be
\braket{O} = \f{\int\psi^*O\psi\prod_i\dd x_i}{\int|\psi|^2\prod_i\dd x_i},
\ee
This is not exactly on the form of equation \ref{exp}, but it can be rewritten in terms of local variables given by
\be
O_L = \f{1}{\psi}O\psi, \label{localdef}
\ee
so that the expectation value is
\be
\braket{O} =  \f{\int|\psi|^2O_L\prod_i\dd x_i}{\int|\psi|^2\prod_i\dd x_i}.
\ee
So the expectation value of the energy is given by
\be
\braket{E} =  \f{\int|\psi|^2E_L\prod_i\dd x_i}{\int|\psi|^2\prod_i\dd x_i},
\ee
with
\be
E_L = \f{1}{\psi}H\psi, \label{localEdef}.
\ee



\subsubsection{2 Particle Case}

In the two particle case with hamiltonean given by equation \ref{ham} and trial wavefunction by equation \ref{2pw} the local energy is given by
\be
E_L(\bb{r_1},\bb{r_2}) = \left[\f{1}{2}(1- \alpha)\omega(r_1^2 + r_2^2) + 2\alpha\omega\right]  - \f{a}{(1+\beta r_{12})^2}\left[\f{a}{(1+\beta r_{12})^2}+  \f{1}{r_{12}} - \f{2\beta}{(1+\beta r_{12})} -\alpha\omega r_{12}\right] + \f{1}{r_{12}},
\ee
as shown in appendix \ref{applap}.

\subsubsection{Slater Case}
In the $N$ particle case with hamiltonean given by equation \ref{ham} and trial wavefunction by equation \ref{npw} the local energy is given by

\be
E_L = -\f{1}{2}\left(\f{\nabla^2|S_\dar|}{|S_\dar|} + \f{\nabla^2|S_\uar|}{|S_\uar|} + \f{\nabla^2J}{J} +  \left(\f{\nabla|S_\dar|}{|S_\dar|} + \f{\nabla|S_\uar|}{|S_\uar|}\right)\cdot
\f{\nabla J}{J}\right) + \f{1}{2}\omega^2\sum_ir_i^2 + \sum_{i<j}\f{1}{r_{ij}},
\ee
where the expressions for the laplacians are derived in appendix \ref{applapn} and are too big to include here.

\subsection{Error Estimation and Blocking}

Let $\Set{F_i}$ be a finite set of consecutive measurements of the quantity $F$ on a system in equilibrium, in other words the probability distribution for the measurements is
independent of time. For a true average of $\braket{F}$ the error of each measurement is given by
\be
e_i = F_i - \braket{F},
\ee
while the error in the mean of the measurements $\bar{F}$ is given by
\be
E = \bar{F} - \braket{F}.
\ee
The variance of $F$ is given by
\be
\sigma^2_F = \braket{(F-\braket{F})^2} = \braket{e^2},
\ee
while the variance of $\bar{F}$ is given by
\be
\sigma^2_{\bar{F}} = \braket{(\bar{F}-\braket{F})^2} = \braket{E^2},
\ee
noting that $E = (\sum_ie_i)/n$ we see that
\be
\sigma^2_{\bar{F}} = \braket{\left(\f{1}{n}\sum_ie_i\right)^2} = \f{1}{n^2}\sum_{i,j}\braket{e_ie_j}.\label{sigmam}
\ee

\subsubsection[Uncorrelated Data]{Uncorrelated Data \footnote{This section is based on the treatment in \cite{errors}}}
For uncorrelated data
\be
 \braket{e_i e_j} = 0,
\ee
when $i\neq j$
and so equation \ref{sigmam} says
\be
\sigma^2_{\bar{F}} = \f{1}{n}\braket{e^2} = \f{1}{n}\sigma_F^2.
\ee
These quantities depend on the true average $\braket{F}$ and are thus unknown, so we instead look at the following known quantity
\be
d_i = F_i - \bar{F} = e_i - E, \label{residualdef}
\ee
with mean square
\be
\bar{d^2} = \f{1}{n}\sum_n(e_i - E)^2 = \f{1}{n}\sum_i e_i^2 - E^2.
\ee
We note that the true average of this is
\be
\braket{\bar{d^2}} = \sigma_F^2 - \sigma_{\bar{F}}^2 = (n-1)\sigma_{\bar{F}}^2,
\ee
or
\be
\sigma_{\bar{F}}^2 = \f{\braket{\bar{d^2}}}{n-1}.
\ee
We still don't know $\braket{\bar{d^2}}$, but we can estimate it with $\bar{d^2}$. And so our estimate of $\braket{E}$ is
\be
\braket{E} = \sigma_{\bar{F}} \approx \sqrt{\f{\bar{d^2}}{n-1}}.\label{errorest}
\ee

If the measurements come with errors the errors propagate as
\be
\sigma^2_{\bar{F}} = \f{1}{n^2}\sum_i\sigma_{F_i}^2.\label{errprop}
\ee

\subsubsection[Correlations and Blocking]{Correlations and Blocking \footnote {This section is based on \cite{block} } }

%%  Then as above the average of $\Set{F_i}$ $\bar{F}$ is an estimate of the true average of $F$ $\braket{F}$. An estimate of the error in this aproximation
%% is given by the standard deviation of $\bar{F}$ $\sigma_{\bar{F}}$. The variance in $\bar{F}$ is given by
%% \be
%% \sigma_{\bar{F}}^2 = \Braket{\bar{F}^2} - \braket{\bar{F}}^2,
%% \ee
%% expanding this we find
%% \begin{align*}
%%   \sigma_{\bar{F}}^2 &= \braket{\left(\f{1}{N}\sum_{i=1}^NF_i\right)^2} - \braket{\f{1}{N}\sum_{i=1}^NF_i}^2\\
%%   &=\f{1}{N^2}\sum_{i=1}^N\sum_{j=1}^N\left(\braket{F_iF_j} - \braket{F_i}\braket{F_j}\right)\\
%%   &= \f{1}{N^2}\sum_{i=1}^N\sum_{j=1}^N\mathrm{cov}(F_i,F_j).
%% \end{align*}
When the data is correlated $\braket{e_ie_j}$ is not generally $0$ and so equation \ref{sigmam} reads
\be
\sigma_{\bar{F}}^2 = \f{1}{n}\braket{e^2} + \f{2}{n^2}\sum_{i,j\neq i}\braket{e_ie_j}
\ee
The time independence of the pdf means that $\braket{e_ie_j}$ should only depend on the relative distance between $i$ and $j$
$t \equiv |i-j|$ \cite{block}. So we can define $C(t) = \braket{e_ie_j}$ and write
\be
\sigma_{\bar{F}}^2 = \f{1}{n}\left[C(0) + 2\sum_{t = 0}^{n-1}\left(1-\f{t}{n}\right)C(t)\right]. 
\ee
For finite correlation time there exists some $t = t_{max}$ for which $C(t>t_{max}) = 0$. If
\be
\sigma_{\bar{F}}^2 = \f{1}{n}\left[C(0) + 2\sum_{t = 0}^{t_{max}}\left(1-\f{t}{n}\right)C(t)\right]. \label{errvar}
\ee
We know how to estimate $C(0)$ from above, but $C(t>0)$ is more complicated.
The blocking method allows us to build $C(t>0)$ into a new $C(0)$ which we can estimate in the regular way.


If we transform the data set $\Set{F_i}$ into another set $\Set{\tilde{F}_i}$, where
\be
\tilde{F}_i = \f{1}{B}\sum_{j=1}^{B}F_{(i-1)B+j},\label{block}
\ee
where $B$ is a whole number called the block size. Call $\tilde{F}_i$ a block.
Expanding $\tilde{C}(0)$ we see that
\begin{align*}
  \tilde{C}(0) &= \braket{\tilde{F}_i^2}-\braket{\tilde{F}_i}^2\\
  &= \f{1}{B^2}\sum_{a,b}^B\braket{\tilde{F}_{i+a}\tilde{F}_{i+b}}-\braket{\tilde{F}_{i+a}}\braket{\tilde{F}_{i+b}}\\
  &= \f{1}{B^2}\sum_{a,b}^BC(|a-b|)\\
  &= \f{1}{B}C(0) + \f{2}{B}\sum_{t=1}^{B-1}\left(1-\f{t}{B}\right)C(t),
\end{align*}
and so
\be
\f{\tilde{C}(0)}{N/B} = \f{1}{N}\left[C(0) +\sum_{t = 1}^{B-1}\left(1-\f{t}{B}\right)C(t)\right],\label{blockvar}
\ee
comparing with equation \ref{errvar} we see that blocking has incorporated part of the error from correlations into $\tilde{C}(0)$! So, for an appropriate block size an improved estimate of the error is
\be
\sigma_{\bar{F}}^2 \approx \f{\sigma_{\bar{\tilde{F}}}^2}{N/B -1}.\label{errest}
\ee
To find the approriate block size we may plot this estimate as a function of block size, and choose a value where this seems to flatten. 

\section{Implementation}
The code is found at \url{https://github.com/khhelland/Fys4411/tree/master/2/src}.

\subsection{Structure}
The code is set up as two classes one for the two particle case and a generalisation for 2,6 or 12 particles.
The 2 particle class i called vmc and the many particles class is called slatervmc. Both classes are initialized with a steplenght, frequency,
variational parameters and a seed for the random number generator. To run the integration both classes have a member function called run, which takes
the number of MC cycles and a blocksize. The classes have public booleans that can be set to determine whether or not to switch on the interaction,
jastrow factor, importance sampling and, in the 2 particle case, numerical differentiation. Both classes have member functions that write the energies to file,
and functions that perform a steepest descent parameter optimization. Due to problems with the automatic steepest descent getting stuck in local minima or becomming unstable
there is a manual steepest descent which uses a static steplength and simply prints updated parameters untill stopped. The idea is to use it by starting it with a relatively large
stepsize, see how it behaves, and then try again with a better initial guess and smaller stepsize.

Due to time constraints the code is somewhat messy, but hopefully it is readable anyway.
Due to the time constraints not alot of time has been put into optimization, and there are several obvious inefficiencies.

\subsection{Paralellization}




\section{Benchmarks}

\subsection{2 Particle Case}
A first test of the code is to check that when the interaction and jastrow factor are turned off and $\alpha = 1$ the energy is 2 with variance 0.

\section{Results}

\begin{table}
  \begin{center}
    \caption{Table of optimal parameters found by steepest descent.}
    \label{paramtab}
    
    \begin{tabular}{*{5}c}
      \toprule
      $N$ & $\omega$ & $\alpha$ (no Jastrow) & $\alpha$ & $\beta$ \\
      \midrule
      2 & 1 & 1 & 1 & 0.4\\
      & 0.5 & 1 & 1 & 0.4\\
      & 0.1 & 1 & 1 & 0.4\\
      & 0.05 & 1 & 1 & 0.4\\
      & 0.01 & 1 & 1 & 0.4\\
        \midrule
      6 & 1 & 1 & 1 & 0.4\\
      & 0.5 & 1 & 1 & 0.4\\
      & 0.1 & 1 & 1 & 0.4\\
      & 0.05 & 1 & 1 & 0.4\\
      & 0.01 & 1 & 1 & 0.4\\
      \midrule
      12 & 1 & 1 & 1 & 0.4\\
      & 0.5 & 1 & 1 & 0.4\\
      & 0.1 & 1 & 1 & 0.4\\
      & 0.05 & 1 & 1 & 0.4\\
      & 0.01 & 1 & 1 & 0.4\\
      \bottomrule
    \end{tabular}
    \end{center}
\end{table}



\begin{table}
  \begin{center}
    \caption{}
    \label{Etab}
    
    \begin{tabular}{*{11}c}
      \toprule
      $N$ & $\omega$ & $\bar{E}$  & $\sigma_{\bar{E}}$ &  $\bar{K}$ & $\bar{V}$& & & & \\
      \midrule
      2 & 1 &  & & & & & & &\\
      & 0.5 &  & & & & & & &\\
      & 0.1 &  & & & & & & &\\
      & 0.05&  & & & & & & &\\
      & 0.01&  & & & & & & &\\
      \midrule
      6 & 1 &  & & & & & & &\\
      & 0.5 &  & & & & & & &\\
      & 0.1 &  & & & & & & &\\
      & 0.05&  & & & & & & &\\
      & 0.01&  & & & & & & &\\
      \midrule
      12 & 1 &  & & & & & & &\\
      & 0.5 &  & & & & & & &\\
      & 0.1 &  & & & & & & &\\
      & 0.05&  & & & & & & &\\
      & 0.01&  & & & & & & &\\
     
      
      \bottomrule
    \end{tabular}
    \end{center}
\end{table}


\subsection{Cost}

\subsection{Accuracy}


\section{Discussion}

\section{Conclusion}



\appendix
\section{A Note About Cofactors and Determinants}\label{cofac}
Determinants can be somewhat hard to work with, following \cite{mortenbok} we may use cofactors to do some trickery.
For a matrix $A$ with elements $A_{ij}$ the cofactors $C_{ij}$ are given by $(-1)^{i+j} M_{ij}$, with $M_{ij}$ being the determinant of the matrix formed by deliting the $i$th
row and $j$th collumn from $A$. There two properties of cofactors that will be important here are
\be
|A|\id = A C^T \Leftrightarrow |A| = \sum_jA_{ij}C_{ij},
\ee
and thus
\be
A^{-1} = \f{1}{|A|}C^T \Leftrightarrow A^{-1}_{ij} = \f{1}{|A|}C_{ji},
\ee
and the fact that $C_{ij}$ is independent of row $i$ and collumn $j$ of A.





\section{Derivatives of the 2 particle trial wavefunction}
In the course of the project we needed analytical expressions for different derivatives of the 2 particle trial wavefunction \ref{2pw}. I have collected
the differentiations here.

\subsection{Gradient}\label{app2pg}
In order to compute the driftforce for importance sampling we needed
\[\f{1}{\psi_T}\pddt{\psi_T}{z},\]
where \(z_i = x_1,x_2,y_1,y_2\).
As \(\psi_T\) is an exponential
\begin{align*}
  \f{1}{\psi_T}\pddt{\psi_T}{z_i} &= \pddt{}{z_i}\left(-\f{1}{2}\alpha\omega(r_1^2+r_2^2) + \f{ar_{12}}{1+\beta r_{12}}\right)\\
  &= -\alpha\omega z_i + a\left(\f{1}{1+\beta r_{12}} - \f{r_{12}\beta}{(1+\beta r_{12})^2}\right)\pddt{r_{12}}{z_i}\\
  &= -\alpha\omega z_i + \f{a}{(1+\beta r_{12})^2}\pddt{r_{12}}{z_i}\\
  &= -\alpha\omega z_i + \f{a}{(1+\beta r_{12})^2}\f{(z_i-z_j)}{r_{12}},\\
\end{align*}
where \(z_j\) is the coordinate of the other particle along the same dimension as \(z_i\), so for example when
\(z_i=x_2\), \(z_j = x_1\).

\subsection{Laplacian}\label{applap}

The local energy is defined as
\[E_L(\bb{r_1},\bb{r_2}) = \f{1}{\psi_T}H\psi_T.\]
In our case the Hamiltonean is
\[H = \sum_{i=1}^2-\f{1}{2}\nabla_i^2+\f{1}{2}\omega^2r_i^2 + \f{1}{r_{12}}.\]
The laplacian is
\[\sum_i\nabla^2 = \sum_i\pndt{}{z_i}{2},\]
where \(z_i\) is the same as above.

Again since \(\psi_T\) is an exponential we have
\begin{align*}
  \f{1}{\psi_T}\pndt{\psi_T}{z_i}{2} &= \pndt{}{z_i}{2}\left(-\alpha\omega(r_1^2+r_2^2)/2 + \f{ar_{12}}{1+\beta r_{12}}\right) +
    \left(\pddt{}{z_i}\left(-\alpha\omega(r_1^2+r_2^2)/2 + \f{ar_{12}}{1+\beta r_{12}}\right)\right)^2\\
    &= \pddt{}{z_i}\left(-\alpha\omega z_i + \f{a}{(1+\beta r_{12})^2}\f{(z_i-z_j)}{r_{12}}\right) +
    \left(-\alpha\omega z_i + \f{a}{(1+\beta r_{12})^2}\f{(z_i-z_j)}{r_{12}}\right)^2\\
    &= -\alpha\omega + \f{a}{(1+\beta r_{12})^2r_{12}} - \f{2a\beta(z_i-z_j)^2}{(1+\beta r_{12})^3r_{12}^2}
    - \f{a(z_i-z_j)^2}{(1+\beta r_{12})^2r_{12}^3} + \alpha^2\omega^2z_i^2 -
    \f{2\alpha\omega a z_i(z_i-z_j)}{(1+\beta r_{12})^2r_{12}} + \f{a^2(z_i-z_j)^2}{(1+\beta r_{12})^4r_{12}^2}.
\end{align*}
So, because the $\psi_T$ is symmetric under $1\leftrightarrow 2$ and $x\leftrightarrow y$:

\[\f{1}{\psi_T}\sum_i\nabla^2_i\psi_T = \alpha\omega(r_1^2 + r_2^2) -4\alpha\omega  + \f{2a^2}{(1+\beta r_{12})^4}+  \f{4a}{(1+\beta r_{12})^2r_{12}} - \f{4a\beta}{(1+\beta r_{12})^3} -
\f{2a}{(1+\beta r_{12})^2r_{12}} -\f{2\alpha\omega a r_{12}}{(1+\beta r_{12})^2}\]
and finally:
\[\f{1}{\psi_T}\sum_i\nabla^2_i\psi_T = \alpha\omega(r_1^2 + r_2^2) -4\alpha\omega  + \f{2a}{(1+\beta r_{12})^2}\left[\f{a}{(1+\beta r_{12})^2}+  \f{1}{r_{12}} - \f{2\beta}{(1+\beta r_{12})} -\alpha\omega r_{12}\right]\]

Using this expression we have the local energy as
\[E_L(\bb{r_1},\bb{r_2}) = \left[\f{1}{2}(1- \alpha)\omega(r_1^2 + r_2^2) + 2\alpha\omega\right]  - \f{a}{(1+\beta r_{12})^2}\left[\f{a}{(1+\beta r_{12})^2}+  \f{1}{r_{12}} - \f{2\beta}{(1+\beta r_{12})} -\alpha\omega r_{12}\right] + \f{1}{r_{12}},\]
where the second term contains the  terms from the  Jastrow-factor and the cross term and the third is the interaction term.

\subsection{Derivatives w.r.t \(\alpha\) and \(\beta\)} \label{app2pab}
In order to find the optimal parameters \(\alpha\) and \(\beta\) we needed the derivatives of \(\psi_T\) with respect to these.
\[  \f{1}{\psi_T}\pddt{\psi_T}{\alpha} = -\f{1}{2}\omega(r_1^2 + r_2^2).\]
\[  \f{1}{\psi_T}\pddt{\psi_T}{\beta} = -\f{ar_{12}^2}{(1+\beta r_{12})^2}.\]

\section{Derivatives of the N particle trial wavefunction}
In the course of the project we needed analytical expressions for different derivatives of the N particle trial wavefunction \ref{npw}. I have collected
the differentiations here.

\subsection{Gradient}\label{appgradn}
We note that for any coordinate $z_i$ only one of the determinants depends on it. So
\[
\frac{1}{\psi_T}\pddt{\psi_T}{z_i} = \f{1}{|s|}\pddt{|s|}{z_i} + \f{1}{J}\pddt{J}{z_i}.
\]
For the derivative of the determinant we exploit the properties noted in section \ref{cofac} to see that
\[
\f{1}{|s|}\pddt{|s|}{z_i} = \f{1}{|s|}\sum_j\pddt{s_{ij}}{z_i}C_{ij},
\]
since $C_{ij}$ is independent of $z_i$. Rewriting $C_{ij} = |s|s^{-1}_{ji}$ we find
\[
\f{1}{|s|}\pddt{|s|}{z_i} = \sum_j\pddt{s_{ij}}{z_i}s^{-1}_{ji}.
\]
From section \ref{app2pg} it is clear that
\[
\f{1}{J}\pddt{J}{z_i} = \sum_{j\neq i}\f{a_{ij}}{(1+\beta r_{ij})^2}\f{(z_i-z_j)}{r_{ij}},
\]
so
\[
\frac{1}{\psi_T}\pddt{\psi_T}{z_i} = \sum_j\pddt{s_{ij}}{z_i}s^{-1}_{ji} + \sum_{j\neq i}\f{a_{ij}}{(1+\beta r_{ij})^2}\f{(z_i-z_j)}{r_{ij}} .
\]


\subsection{Laplacian}\label{applapn}

Writing $\psi_T = \psi_SJ$ we can write the Laplacian as
\[
\f{\nabla^2\psi_T}{\psi_T} = \f{\nabla^2\psi_S}{\psi_S} + \f{\nabla^2J}{J} + 2\f{\nabla\psi_S}{\psi_S}\cdot\f{\nabla J}{J},
\]
expanding the Slater part $\psi_S =|S|_{\dar}|S|_{\uar}$ we find
\[
\f{\nabla^2\psi_T}{\psi_T} = \f{\nabla^2|S|_{\dar}}{|S|_{\dar}} + \f{\nabla^2|S|_{\uar}}{|S|_{\uar}} + \f{\nabla^2J}{J}
+ 2\left(\f{\nabla |S|_{\dar}}{|S|_{\dar}} + \f{\nabla |S|_{\uar}}{|S|_{\uar}}\right)\cdot\f{\nabla J}{J},
\]
noting that there is no cross-term between the determinants as they depend on different coordinates.
By the same arguments as above we find
\[
\f{1}{|S|}\pndt{|S|}{z_i}{2} = \sum_j \pndt{s_{ij}}{z_i}{2}s^{-1}_{ji},
\]
which means that
\begin{align*}
  \f{\nabla^2\psi_s}{\psi_s} &= \sum_{ij}(\nabla^2s^\dar_{ij})s^{\dar,-1}_{ji} + (\nabla^2s^\uar_{ij})s^{\uar,-1}_{ji}\\
  &= \sum_{ij}^{N/2}(\nabla^2\psi_{\dar,j}(\bb{r}_i))s^{\dar,-1}_{ji} + (\nabla^2\psi_{\uar,j}(\bb{r}_i))s^{\uar,-1}_{ji} \\
  &= \sum_{ij}^{N/2}(\alpha^2\omega^2r^2_i-2\alpha\omega(n_{x,j} + n_{y,j} + 1))\psi_j(\bb{r_i})s^{\dar,-1}_{ji} +
  (\alpha^2\omega^2r^2_{i + N/2}-2\alpha\omega(n_{x,j} + n_{y,j} + 1))\psi_j(\bb{r_{i+N/2}})s^{\uar,-1}_{ji}\\
  &= \sum_{ij}^{N/2}(\alpha^2\omega^2r^2_i-2\alpha\omega(n_{x,j} + n_{y,j} + 1))s^{\dar}_{ij}s^{\dar,-1}_{ji} +
  (\alpha^2\omega^2r^2_{i + N/2}-2\alpha\omega(n_{x,j} + n_{y,j} + 1))s^{\uar}_{ij}s^{\uar,-1}_{ji}\\
  &=\sum_{i=1}^{N}\alpha^2\omega^2r^2_{i} -4\alpha\omega\sum_{j}^{N/2}(n_{x,j} + n_{y,j} + 1),
\end{align*}
where I have used that $\sum_j s_{ij}s^{-1}_{ji} = \sum_i s_{ji}^{-1}s_{ij} = 1$.
For the jastrow part
\[
\f{1}{J}\pndt{J}{z_i}{2} = \sum_{j\neq i}\pndt{f_{ij}}{z_i}{2} + \left(\sum_{j\neq i}\pddt{f_{ij}}{z_i}\right)^2,
\]
with $f_{ij} = a_{ij}r_{ij}/(1+\beta r_{ij})$, so
\[
\f{1}{J}\pndt{J}{z_i}{2} = \sum_{j \neq i}\f{a_{ij}}{(1+\beta r_{ij})^2}\left[\f{1}{r_{ij}}\left(1 -\left(\f{(z_i-z_j)}{r_{ij}}\right)^2\right) -\f{2\beta}{(1+\beta r_{ij})}\left(\f{(z_i-z_j)}{r_{ij}}\right)^2\right] + \left(\sum_{j\neq i}\pddt{f_{ij}}{z_i}\right)^2,
\]
and
\[
\f{\nabla^2_iJ}{J} = \sum_{j \neq i}\f{a_{ij}}{(1+\beta r_{ij})^2}\left[\f{1}{r_{ij}} -\f{2\beta}{(1+\beta r_{ij})}\right]
+ \left(\sum_{j\neq i}\pddt{f_{ij}}{x_i}\right)^2 + \left(\sum_{j\neq i}\pddt{f_{ij}}{y_i}\right)^2,
\]
and finally
\[
\f{\nabla^2J}{J} =2 \sum_{j < i}\f{a_{ij}}{(1+\beta r_{ij})^2}\left[\f{1}{r_{ij}} -\f{2\beta}{(1+\beta r_{ij})}\right]
+ \sum_i \left[\left(\sum_{j\neq i}\pddt{f_{ij}}{x_i}\right)^2 + \left(\sum_{j\neq i}\pddt{f_{ij}}{y_i}\right)^2\right],
\]
The cross terms become
\[
 2\left(\f{\nabla |S|_{\dar}}{|S|_{\dar}} + \f{\nabla |S|_{\uar}}{|S|_{\uar}}\right)\cdot\f{\nabla J}{J} = 2\sum_i\left(\f{1}{|s|}\pddt{|s|}{x_i}\f{1}{J}\pddt{J}{x_i} + \f{1}{|s|}\pddt{|s|}{y_i}\f{1}{J}\pddt{J}{y_i}\right). 
\]

\subsection{Derivative w.r.t \(\alpha\) and \(\beta\)}
The many particle generalizations of the formulas from section \ref{app2pab}.
For $\alpha$ we again need the derivative of a determinant, but the arguments for the adjugate $C^T$ being independent of the diffentiation variable used above is no longer valid.
Luckily Jacobis formula says that we may still use a similar expression:
\[
\f{1}{|A|}\ddt{|A|}{\alpha} = \sum_{i,j}\ddt{A_{ij}}{\alpha}A^{-1}_{ji}.
\]
In our case we then have
\[
\f{1}{\psi_T}\pddt{\psi_T}{\alpha} = \sum_{i,j}\pddt{\psi_i(\bb{r}_j)}{\alpha}s^{\dar,-1}_{ji} + \pddt{\psi_i(\bb{r}_{j+N/2})}{\alpha}s^{\uar,-1}_{ji}. 
\]
\begin{align*}
  \pddt{\psi_i(\bb{r})}{\alpha} &= \omega \pddt{\psi_i(\bb{r})}{\alpha\omega}\\
  &= \omega \left(\pddt{\sqrt{\alpha\omega}x}{\alpha}H'_{n_x}(\sqrt{\alpha\omega}x)H_{n_y}(\sqrt{\alpha\omega}y)
  + \pddt{\sqrt{\alpha\omega}y}{\alpha}H_{n_x}(\sqrt{\alpha\omega}x)H'_{n_y}(\sqrt{\alpha\omega}y)\right.\\
  &\qquad\qquad\left.-\f{1}{2}(x^2 + y^2) H_{n_x}(\sqrt{\alpha\omega}x)H_{n_y}(\sqrt{\alpha\omega}y)\right)e^{-\alpha\omega(x^2 + y^2)/2}\\
  &= \omega \left(\f{xn_x}{\sqrt{\alpha\omega}}H_{n_x-1}(\sqrt{\alpha\omega}x)H_{n_y}(\sqrt{\alpha\omega}y)
  + \f{yn_y}{\sqrt{\alpha\omega}}H_{n_x}(\sqrt{\alpha\omega}x)H_{n_y-1}(\sqrt{\alpha\omega}y)\right.\\
  &\qquad\qquad\left.-\f{1}{2}(x^2 + y^2) H_{n_x}(\sqrt{\alpha\omega}x)H_{n_y}(\sqrt{\alpha\omega}y)\right)e^{-\alpha\omega(x^2 + y^2)/2}.
\end{align*}

The $\beta$ derivative is a lot simpler: 

\[\f{1}{\psi_T}\pddt{\psi_T}{\beta} = \sum_{i<j}\pddt{f_{ij}}{\beta} = \sum_{i<j} -\f{a_{ij}r_{ij}^2}{(1+\beta r_{ij})^2}.\]





\begin{thebibliography}{1}

\bibitem{proj1}
  K. Helland ``Fys4411: Project 1'' available at \url{https://github.com/khhelland/Fys4411/blob/master/1/rapport.pdf}
  
\bibitem{errors}
  G.L. Squires \emph{Practical Physics}, 4th edition. Cambridge University Press,2001.
  
\bibitem{block}
  H. Flyvbjerg and H.G. Petersen ``Error estimates on averages of correlated data''. The Journal of Chemical Physics 91, 461, 1989

\bibitem{mortenbok}
  M. Hjort-Jensen, \emph{Computational Physics: Lecture Notes Fall 2015}, found at
  \url{https://github.com/CompPhysics/ComputationalPhysics2/blob/gh-pages/doc/Literature/lectures2015.pdf}
  on the 31/01/17
\bibitem{mortenslides}
  M. Hjort-Jensen,\emph{Computational Physics: Hartree-Fock methods and introduction to Many-body Theory}
  found at \url{http://compphysics.github.io/ComputationalPhysics2/doc/pub/basicMB/pdf/basicMB-print.pdf}
  on the 16/03/17
%% \bibitem{simen}
%%   Simen Kvaal, \emph{Lecture notes for Fys-Kjm4480: Quantum Mechanics for many-particle systems},
%%   october 2016, not published.
\end{thebibliography}


%% \begin{table}
%%   \caption{}\label{kal1}
%%   \begin{center}
%%     \begin{tabular}{|r|r|}
      
%%       \hline
%%       \textbf{masse/g}& \textbf{Utslag/mm}\\ \hline
%%     \end{tabular}
%%   \end{center}
%% \end{table}
      




%% \begin{figure}
%%   \begin{center}
%%     \fbox{\includegraphics[width = 12cm]{}}
%%   \end{center}
%%   \caption{}\label{}
%% \end{figure}


\end{document}
