\documentclass[a4paper,norsk,10pt]{article}
\usepackage[utf8]{inputenc}
\usepackage[T1]{fontenc}
\usepackage[norsk]{babel}
\usepackage{graphicx,mathpple, textcomp, varioref}
\usepackage{fullpage}
\usepackage{fancyhdr}
\usepackage{lastpage}
\usepackage{hyperref}
\usepackage{amsmath}
\usepackage{braket}
\usepackage{enumitem}

\title{Fys4110: Project 2}
\author{Knut Halvor Helland}
\pagestyle{fancyplain}
\fancyhf{}
\renewcommand{\headrulewidth}{0pt}
\fancyfoot[R]{\thepage/\pageref{LastPage}}
\tolerance = 5000
\hbadness = \tolerance
\pretolerance = 2000
\setlength{\headheight}{20pt}

\newcommand{\unit}[1]{\; \mathrm{#1}}
\newcommand{\bb}[1]{\boldsymbol{#1}}
\newcommand{\p}{\partial}
\newcommand{\dd}{\mathrm{d}}
\newcommand{\ddt}[2]{\frac{\dd #1}{\dd #2}}
\newcommand{\dndt}[3]{\frac{\dd^{#3} #1}{\dd #2^{#3}}}
\newcommand{\pddt}[2]{\frac{\p #1}{\p #2}}
\newcommand{\pndt}[3]{\frac{\p^{#3} #1}{\p #2^{#3}}}
\newcommand{\Rar}{\Rightarrow}
\newcommand{\rar}{\rightarrow}
\newcommand{\lagr}{\mathscr{L}}
\newcommand{\ham}{\mathcal{H}}
\newcommand{\id}{\bb{1}}
\newcommand{\deldt}[2]{\frac{\delta #1}{\delta #2}}
\newcommand{\be}{\begin{equation}}
\newcommand{\ee}{\end{equation}}
\newcommand{\f}{\frac}


\renewcommand{\bar}{\overline}


\begin{document}
\maketitle{}
\begin{abstract}
$\ldots$
\end{abstract}


\section{Introduction}
In this project I continue the study of many interacting particles in an isotropic two dimensional harmonic oscillator.
In project 1 \cite{proj1} I used Hartree-Fock methods to construct the Slater determinant of linear combinations of single
particle non-interacting states that minimized the energy. In this project I would like to improve this estimate of the ground state
by adding a Jastrow factor. Then the energy cannot be found with Hartree-Fock methods, but rather from direct integration of the Hamiltonean.
I will do this using the Metropolis algorithm to pick out integration points.
I will add a Jastrow factor with a free parameter, minimize the energy with respect to this parameter, and
then calculate the energy of this new approximation of the ground state.
In addition I will study the 2 particle case with a singly parametrized symmetric wavefunction instead of the Slater determinant.


\section{Physical Problem}
The full Hamiltonean of the problem with $N$ particles and using atomic units is
\be
H = \sum_i^N -\f{1}{2}\nabla^2_i + \f{1}{2}\omega^2r^2_i + \sum_{i<j}^N\f{1}{r_{ij}}, \label{ham}
\ee
where $r_{ij} = |\bb{r}_i-\bb{r}_j|$.

\subsection{2 Particle Case}
For the two particle case I will aproximate the ground state with the parametrized trial wavefunction
\be
\psi_T = A\exp\left(-\alpha\omega(r_1^2 + r_2^2) -\f{r_{12}}{1+\beta r_{12}}\right),\label{2pw}
\ee
with $\alpha$ and $\beta$ as free parameters.

\subsection{Slater case}
For the many particle case we will use a Slater determinant constructed with Hartree-Fock methods as described in \cite{proj1}.
We will multiply this with a Jastrow factor with one free parameter.

\be
\psi_T = A S\prod_{i<j}^N\exp\left(-\f{a_{ij}r_{ij}}{1+\beta r_{ij}}\right),\label{npw}
\ee
where $S$ is the Slater determinant, and $a_{ij}$ is 1 when the particles have anti-paralell spins and $1/3$ when paralell.

\section{Variational Monte-Carlo}


% wavefunction, slater, jastrow



\section{The Metropolis Algorithm}

The Metropolis algorithm is an algorithm for computing expectation values for functions of many variables efficiently.
It is based on semi-randomly walking through the integration space and finding the value of the function at each point.
Let \(F(x_1,\ldots,x_N)\) be the function we wish to find the expectation value of. Then
\be
\braket{F} = \f{\int F(x_1,\ldots,x_N)p(x_1,\ldots,x_N)\prod_{k=1}^N\dd x_k}{\int p(x_1,\ldots,x_N)\prod_{k=1}^N\dd x_k}, \label{exp}
\ee
where \(p(x_1,\ldots,x_N)\) is the (generally) non-normalized probability density function. In our case \(p\) is \(|\psi|^2\).
One advantage with this method is that it does not require explicit normalization of the probalility density.



\subsection{Detailed Balance}
The Metropolis algorithm may be derived by demanding that the Markov chain exhibits detailed balance.
The criterion for detailed balance is
\be
P(a)P(b|a) = P(b)P(a|b),
\ee
or rewritten
\be
\f{P(a)}{P(b)} = \f{P(a|b)}{P(b|a)}.
\ee
We may split \(P(a|b) = G(a|b)A(a|b)\), where \(G(a|b)\) is the probability of proposing a move from $b$ to $a$, while
$A(a|b)$ is the probability of accepting a proposed move from $b$ to $a$. The the detailed balance requirement may be rewritten as
\be
\f{A(a|b)}{A(b|a)} = \f{P(a)G(b|a)}{P(b)G(a|b)}.
\ee
Now we choose an acceptance ratio that satisfies this requirement. The Metropolis choice is to use
\be
A(a|b) = \mathrm{min}\left(1,\f{P(a)G(b|a)}{P(b)G(a|b)}\right).\label{arat}
\ee

So the Metropolis algorithm for drawing points from a probability distrobution is
\begin{enumerate}
\item
  Draw a proposed move from the proposal distribution.
\item
  Evaluate the acceptance ratio $a$ from equation \ref{arat}.
\item
  Draw a random number $0 \leq r < 1$ from a uniform distribution.
\item
  If $a>r$ accept the move. Else reject the move.
\item
  Save position
\item
  return to point 1.
\end{enumerate}
  
  

\subsection{Symmetric Proposal Density}

If the proposal probality $G(a|b) = G(b|a)$ the proposal distribution is called symmetric and drops out of the acceptance ratio.
Thus there is less to calculate for each loop in the algortihm. However symmetric proposal densities lead to many proposed steps being rejected,
and thus to higher corelations between the points.


\subsection{Importance Sampling}

If we instead choose a non-symmetric proposal distribution we may choose one to maximise the acceptancerate.
If we choose a proposaldistribution so that the probability of proposing a move into an area with a higher probability is higher than
the probability of proposing a move into an area with a lower probability. In other words
if $P(a)>P(b)$ then $G(a|b)>G(b|a)$.

\subsection{Local Energy}
In quantum mechanics the probability distribution is given by $|\psi|^2$ and the expectation value for an operator $O$ is given by
\be
\braket{O} = \f{\int\psi^*O\psi\prod_i\dd x_i}{\int|\psi|^2\prod_i\dd x_i},
\ee
This is not exactly on the form of equation \ref{exp}, but it can be rewritten in terms of local variables given by
\be
O_L = \f{1}{\psi}O\psi, \label{localdef}
\ee
so that the expectation value is
\be
\braket{O} =  \f{\int|\psi|^2O_L\prod_i\dd x_i}{\int|\psi|^2\prod_i\dd x_i}.
\ee
So the expectation value of the energy is given by
\be
\braket{E} =  \f{\int|\psi|^2E_L\prod_i\dd x_i}{\int|\psi|^2\prod_i\dd x_i},
\ee
with
\be
E_L = \f{1}{\psi}H\psi, \label{localEdef}.
\ee
In the two particle case with hamiltonean given by equation \ref{ham} and trial wavefunction by equation \ref{2pw} the local energy is given by
\be
E_L(\bb{r_1},\bb{r_2}) = \left[\f{1}{2}(1- \alpha)\omega(r_1^2 + r_2^2) + 2\alpha\omega\right]  - \f{a}{(1+\beta r_{12})^2}\left[\f{a}{(1+\beta r_{12})^2}+  \f{1}{r_{12}} - \f{2\beta}{(1+\beta r_{12})} -\alpha\omega r_{12}\right] + \f{1}{r_{12}},
\ee
as shown in appendix \ref{applap}.
\section{Implementation}

\subsection{Structure}
\subsection{Paralellization}

\section{Cost}

\section{Accuracy}

\subsection{Blocking}

\section{Benchmarks}

\subsection{2 Particle Case}
A first test of the code is to check that when the interaction and jastrow factor are turned off and $\alpha = 1$ the energy is 2 with variance 0.

\section{Results}

\section{Discussion}

\section{Conclusion}



\appendix
\section{Derivatives of the 2 particle trial wavefunction}
In the course of the project we needed analytical expressions for different derivatives of the 2 particle trial wavefunction. I have collected
the differentiations here.
Trial wave function is
\[\psi_T(\bb{r_1},\bb{r_2}) = C\exp\left(-\alpha\omega(r_1^2+r_2^2)/2\right)\exp\left(\f{ar_{12}}{1+\beta r_{12}}\right).\]

\subsection{Gradient}
In order to compute the driftforce for importance sampling we needed
\[\f{1}{\psi_T}\ddt{\psi_T}{z},\]
where \(z_i = x_1,x_2,y_1,y_2\).
As \(\psi_T\) is an exponential
\begin{align*}
  \f{1}{\psi_T}\ddt{\psi_T}{z_i} &= \ddt{}{z_i}\left(-\f{1}{2}\alpha\omega(r_1^2+r_2^2) + \f{ar_{12}}{1+\beta r_{12}}\right)\\
  &= -\alpha\omega z_i + a\left(\f{1}{1+\beta r_{12}} - \f{r_{12}\beta}{(1+\beta r_{12})^2}\right)\pddt{r_{12}}{z_i}\\
  &= -\alpha\omega z_i + \f{a}{(1+\beta r_{12})^2}\pddt{r_{12}}{z_i}\\
  &= -\alpha\omega z_i + \f{a}{(1+\beta r_{12})^2}\f{(z_i-z_j)}{r_{12}},\\
\end{align*}
where \(z_j\) is the coordinate of the other particle along the same dimension as \(z_i\), so for example when
\(z_i=x_2\), \(z_j = x_1\).

\subsection{Laplacian}\label{applap}

The local energy is defined as
\[E_L(\bb{r_1},\bb{r_2}) = \f{1}{\psi_T}H\psi_T.\]
In our case the Hamiltonean is
\[H = \sum_{i=1}^2-\f{1}{2}\nabla_i^2+\f{1}{2}\omega^2r_i^2 + \f{1}{r_{12}}.\]
The laplacian is
\[\sum_i\nabla^2 = \sum_i\pndt{}{z_i}{2},\]
where \(z_i\) is the same as above.

Again since \(\psi_T\) is an exponential we have
\begin{align*}
  \f{1}{\psi_T}\pndt{\psi_T}{z_i}{2} &= \pndt{}{z_i}{2}\left(-\alpha\omega(r_1^2+r_2^2)/2 + \f{ar_{12}}{1+\beta r_{12}}\right) +
    \left(\pddt{}{z_i}\left(-\alpha\omega(r_1^2+r_2^2)/2 + \f{ar_{12}}{1+\beta r_{12}}\right)\right)^2\\
    &= \pddt{}{z_i}\left(-\alpha\omega z_i + \f{a}{(1+\beta r_{12})^2}\f{(z_i-z_j)}{r_{12}}\right) +
    \left(-\alpha\omega z_i + \f{a}{(1+\beta r_{12})^2}\f{(z_i-z_j)}{r_{12}}\right)^2\\
    &= -\alpha\omega + \f{a}{(1+\beta r_{12})^2r_{12}} - \f{2a\beta(z_i-z_j)^2}{(1+\beta r_{12})^3r_{12}^2}
    - \f{a(z_i-z_j)^2}{(1+\beta r_{12})^2r_{12}^3} + \alpha^2\omega^2z_i^2 -
    \f{2\alpha\omega a z_i(z_i-z_j)}{(1+\beta r_{12})^2r_{12}} + \f{a^2(z_i-z_j)^2}{(1+\beta r_{12})^4r_{12}^2}.
\end{align*}
So, because the $\psi_T$ is symmetric under $1\leftrightarrow 2$ and $x\leftrightarrow y$:

\[\f{1}{\psi_T}\sum_i\nabla^2_i\psi_T = \alpha\omega(r_1^2 + r_2^2) -4\alpha\omega  + \f{2a^2}{(1+\beta r_{12})^4}+  \f{4a}{(1+\beta r_{12})^2r_{12}} - \f{4a\beta}{(1+\beta r_{12})^3} -
\f{2a}{(1+\beta r_{12})^2r_{12}} -\f{2\alpha\omega a r_{12}}{(1+\beta r_{12})^2}\]
and finally:
\[\f{1}{\psi_T}\sum_i\nabla^2_i\psi_T = \alpha\omega(r_1^2 + r_2^2) -4\alpha\omega  + \f{2a}{(1+\beta r_{12})^2}\left[\f{a}{(1+\beta r_{12})^2}+  \f{1}{r_{12}} - \f{2\beta}{(1+\beta r_{12})} -\alpha\omega r_{12}\right]\]

Using this expression we have the local energy as
\[E_L(\bb{r_1},\bb{r_2}) = \left[\f{1}{2}(1- \alpha)\omega(r_1^2 + r_2^2) + 2\alpha\omega\right]  - \f{a}{(1+\beta r_{12})^2}\left[\f{a}{(1+\beta r_{12})^2}+  \f{1}{r_{12}} - \f{2\beta}{(1+\beta r_{12})} -\alpha\omega r_{12}\right] + \f{1}{r_{12}},\]
where the second term contains the  terms from the  Jastrow-factor and the cross term and the third is the interaction term.

\subsection{Derivatives w.r.t \(\alpha\) and \(\beta\)}
In order to find the optimal parameters \(\alpha\) and \(\beta\) we needed the derivatives of \(\psi_T\) with respect to these.
\[  \f{1}{\psi_T}\pddt{\psi_T}{\alpha} = -\f{1}{2}\omega(r_1^2 + r_2^2).\]
\[  \f{1}{\psi_T}\pddt{\psi_T}{\beta} = -\f{ar_{12}^2}{(1+\beta r_{12})^2}.\]

  
%% \begin{table}
%%   \caption{}\label{kal1}
%%   \begin{center}
%%     \begin{tabular}{|r|r|}
      
%%       \hline
%%       \textbf{masse/g}& \textbf{Utslag/mm}\\ \hline
%%     \end{tabular}
%%   \end{center}
%% \end{table}
      




%% \begin{figure}
%%   \begin{center}
%%     \fbox{\includegraphics[width = 12cm]{}}
%%   \end{center}
%%   \caption{}\label{}
%% \end{figure}


\end{document}
