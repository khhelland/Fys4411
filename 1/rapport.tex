\documentclass[a4paper,english,12pt]{article}
\usepackage[utf8]{inputenc}
\usepackage[T1]{fontenc}
\usepackage[english]{babel}
\usepackage{graphicx,mathpple, textcomp, varioref}
\usepackage{fullpage}
\usepackage{fancyhdr}
\usepackage{lastpage}
\usepackage{hyperref}
\usepackage{amsmath}
\usepackage{braket}
\usepackage{enumitem}
\usepackage{algpseudocode,algorithmicx}


\title{Fys4411: Project 1}
\author{Knut Halvor Helland}
\pagestyle{fancyplain}
\fancyhf{}
\renewcommand{\headrulewidth}{0pt}
\fancyfoot[R]{\thepage/\pageref{LastPage}}
\tolerance = 5000
\hbadness = \tolerance
\pretolerance = 2000
\setlength{\headheight}{20pt}

\newcommand{\unit}[1]{\; \mathrm{#1}}
\newcommand{\bb}[1]{\boldsymbol{#1}}
\newcommand{\p}{\partial}
\newcommand{\dd}{\mathrm{d}}
\newcommand{\ddt}[2]{\frac{\dd #1}{\dd #2}}
\newcommand{\pddt}[2]{\frac{\p #1}{\p #2}}
\newcommand{\Rar}{\Rightarrow}
\newcommand{\rar}{\rightarrow}
\newcommand{\lagr}{\mathcal{L}}
\newcommand{\id}{\bb{1}}
\newcommand{\be}{\begin{equation}}
\newcommand{\ee}{\end{equation}}
\newcommand{\f}{\frac}
\newcommand{\deldt}[2]{\frac{\delta #1}{\delta #2}}

\renewcommand{\epsilon}{\varepsilon}

\begin{document}
\maketitle{}

\begin{abstract}
  In this project I study many particles in an isotropic two dimensional quantum harmonic oscillator with the Hartree-Fock method for aproximating
  the ground state.
  I have used subsets of an orthogonal basis \(\Set{\Ket{\phi}}\) of eigenstates of the one particle non-interacting part of the hamiltonean
  as a starting basis. I then find another basis given by \(\ket{\psi_i} = \sum_\alpha C_{i\alpha}\ket{\phi_\alpha}\). I vary the coeficients $C$
  numerically
  to find the Slater determinant $\Ket{\Psi}$ of N $\ket{\psi}$ states that gives the mininal energy for $\Bra{\Psi}H\Ket{\Psi}$. I do this
  for N=2,6,12,20 particles and frequency of the oscillator $\omega = 1,0.5$. For N = 2 and 6 the Hartree-Fock limit is reached for both frequencies,
  for N = 12 and 20 I did not find the limit, but figures \ref{figshellsw1} and \ref{figshellsw05} seem to show that the final calculations were not
  very far from the limit. So it seems that I have found aproximations to the ground state that are at or close to the limit of how well
  the Hartree-Fock method ca do. Still we do not know how good these aproximations are to the true ground state.
  

\end{abstract}

\section{Introduction}

In this project I study a system of many interacting electrons in a two dimensional quantum harmonic oscillator with Hartree-Fock theory.
I will use Hartree-Fock theory to construct a set of single particle states $\ket{\psi}$ from which a Slater determinant $\Ket{\Psi}$
may be constructed. This Slater determinant is then an approximation of the ground state of the system. I find the $\ket{\psi}$s by considering
a basis of eigenstates of the one-particle non-interacting case, and then apply a unitary transformation of this basis. The unitary transformation
is chosen based on varying the coeficients through the Hartree-Fock algorithm, and minimizing the energy.

\section{Physical Problem}

The single particle hamiltonean for an electron in a two dimensional harmonic oscillator is
\[
h_0 = -\f{1}{2}\nabla^2 + \f{1}{2}\omega^2 r^2,
\]
where we are using atomic units where $\hbar = m_e = e = 1$.
This hamiltonean has eigenvalues
\(\ket{\phi_{nm\sigma}}\) with energies \(\epsilon_{nm} = \omega(2n+|m|+1)\), where $n = 0,1,2,\ldots; m = -n,-n+1,\ldots; \sigma = \uparrow,\downarrow$.
This means that there are energy levels with \(\epsilon_i = \omega i\) with degeneracy \(d(i) = i\), where $i$ is a natural number.


The simplest way of adding particles to the potential is to assume that the particles are non-interacting. Then the $N$ particle hamiltonean is simply
\be
H_0 = \sum_{i=1}^N -\f{1}{2}\nabla_i^2 + \f{1}{2}\omega^2 r_i^2.
\ee
Obviously products of eigenstatets of $h_0$ are eigenstates of $H_0$, but these are electrons so the total wavefunction must be antisymmetric under particle exchange.
To account for this we can make a Slater determinant, which is the antisymmetrized product of states. The energy of such a Slater determinant is just the sum of energies from the
individual one particle states. Thus the ground state $\Ket{\Phi}$ of $H_0$ is the Slater determinant of the $N$ lowest energy states $\ket{\phi_{nm\sigma}}$. If $N$ is a magic number, in other words, if $N$ corresponds to $n$ filled shells, then the energy of the $\Ket{\Phi}$ is
\be\label{eqH0}
E[\Phi] = \sum_i^n\epsilon_id(i) = \sum_i^n2i^2\omega = \f{n(n+1)(2n+1)}{3}\omega.
\ee
In particular the ground state energies for $N = 2,6,12$ and $20$ is found in table \ref{tab0}.


\begin{table}
  \caption{Table of ground state energies for non-interacting fermions in isotropic 2D harmonic oscillator}\label{tab0}
  \begin{center}
    \begin{tabular}{*{2}{c}}
      Particles &  $E[\Phi]$ [$\omega$] \\
      \hline
      2&2\\
      6&10\\
      12&28\\
      20&60\\
    \end{tabular}
  \end{center}
\end{table}



Of course we know that electrons interact via the Coulomb interaction, so aproximating electrons as being non-interacting is wrong,
and depending on the strength of the oscillator may be very wrong.
The hamiltonean for many electrons with Coulomb interactions is given by
\be
H = H_0 + H_I,
\ee
where
\be\label{eqHi}
H_I = \sum_{i<j}^N\f{1}{r_{ij}},
\ee
where the sum is over all pairs of distinct particles and $r_{ij} = |\bb{r}_i-\bb{r_j}|$.
Now there is no simple way to find the exact ground state, but we will try to find an approximation.





\section{Hartree-Fock theory}


The variational principle is that for any state $\ket{\psi}$
\be\label{var}
E_{ground}\leq \bra{\psi}H\ket{\psi},
\ee
in other words the expectation value of the energy is always greater than or equal to the ground state energy.
Thus if one could examine the whole Hilbert space and find the state with the lowest energy expectation that state would be the ground state.
This is however easier said than done. Instead we can parameterize a class of functions and minimize the energy with respect to these parameters.

In many cases $H$ can be written as $H = H_0 + H_I$, where $H_0$ is a non-interacting part and
\[H_I =\sum_{i<j}v(r_{ij})\]
is an interacting part composed of two-body interactions.
By analogy to the non-interacting case we may guess that the ground state of $H$ is a Slater determinant. Then the problem is to find out which
single particle states to put into the determinant. It is not unreasonable to guess that these states should have something in common with
the states eigenstates of the non-interacting part of the hamiltonean \(\Set{\Ket{\phi}}\). The full set is an orthonormal basis for the whole Hilbert space,
so any single particle state is a linear combination of $\ket{\phi}$s. We define new states
\be  \ket{\psi_p} = \sum_\alpha C_{p\alpha}\ket{\phi_\alpha}.\ee
We choose $C$ to be unitary. In that case we have found a new basis \(\Set{\Ket{\psi}}\) that is also orthonormal:
\begin{align*}
  \braket{\psi_p|\psi_q} &= \sum_\alpha\braket{\psi_p|\phi_\alpha}\braket{\phi_\alpha|\psi_q}\\
  &=\sum_{\alpha}C_{p\alpha}^*C_{q\alpha}\\
  &=\delta_{pq},
\end{align*}  
where the last step depends on $C$ being unitary.

So we want a Slater determinant of $N$ states from \(\Set{\Ket{\psi}}\), we want to find the unitary matrix $C$ and choice of $N$ $\ket{\psi}$s that minimize the
energy. Since \(\Set{\Ket{\phi}}\) spans the Hilbert space this should in principle give the ideal \(\Set{\Ket{\psi}}\), but since \(\Set{\Ket{\phi}}\)
is infinite we will in practice only use a subset.

%% The Slater determinant $\Psi$ constructed from $\Set{\Ket{\psi}}$ is the determinant of the matrix that may be written in index notation as
%% \[  a_{ij} = \psi_i(\bb{r}_j)
%% =\sum_\alpha C_{i\alpha}\phi_\alpha(\bb{r}_j) = \sum_\alpha C_{i\alpha}b_{\alpha j},
%% \]

%% but the final expression is the product of $C$ and the matrix in the Slater determinant $\Phi$ constructed from $\Set{\Ket{\psi}}$! Since the determinant of a product
%% is the product of the determinants
%% \[\Psi = \det(C)\Phi,\]
%% where $\det(C)$ is just a phase, since $C$ is unitary.

%% These determinants are constructed from the whole bases, but we are instead interested in determinants constructed from the N lowest energy states
%% (by the lowest energy states in $\Set{\Ket{\phi}}$ i mean those with lowest eigenvalue for $h_0$,
%% I will come back to what is meant by lowest energy for the states in $\Set{\Ket{\psi}}$). In this case the
%% relation between the determinants is more complicated, and in particular they may have different energies.

We order \(\Set{\Ket{\psi}}\) and  \(\Set{\Ket{\phi}}\) so that the lowest energy states come first.
\footnote{By the lowest energy states in $\Set{\Ket{\phi}}$ I mean those with lowest eigenvalue for $h_0$,
I will come back to what is meant by lowest energy for the states in $\Set{\Ket{\psi}}$.}
The energy expectation value of a determinant $\Ket{\Psi}$ constructed from the $n$ first states is
\be\label{Ei} E[\Psi] = \bra{\Psi}H\ket{\Psi} = \sum_{i\leq N}\bra{i}h_0\ket{i} + \frac{1}{2}\sum_{ij\leq N}\bra{ij}v\ket{ij}_{AS}.\ee
Rewriting this in terms of $\Set{\Ket{\phi}}$ we obtain
\be\label{Ea}E[\Psi] = \sum_{i\leq N}\sum_{\alpha\beta}C_{i\alpha}^*C_{i\beta}\bra{\alpha}h_0\ket{\beta} +
\frac{1}{2}\sum_{ij\leq N}\sum_{\alpha\beta\gamma\delta}C_{i\alpha}^*C_{j\beta}^*C_{i\gamma}C_{j\delta}\bra{\alpha\beta}v\ket{\gamma\delta}_{AS}.\ee
We want to minimize this energy with respect to $C$ under the constraint $\braket{i|j} = \delta_{ij}$. Introducing the Lagrange multipliers $\lambda_{ij}$
the functional to be minimized is
\begin{align*}
  \lagr &= E[\Psi] - \sum_{ij}^N\lambda_{ij}(\braket{i|j} - \delta_{ij})\\
  &=  E[\Psi] - \sum_{ij}^N\sum_{\alpha\beta}\lambda_{ij}(C_{i\alpha}^*C_{j\beta}\braket{\alpha|\beta} - \delta_{ij})\\
  &= E[\Psi] - \sum_{ij}^N\sum_{\alpha}\lambda_{ij}(C_{i\alpha}^*C_{j\alpha} - \delta_{ij})
\end{align*} 
Now we demand
\be\label{lag}  \pddt{\lagr}{C^*_{p\xi}} = \pddt{E[\Psi]}{C^*_{p\xi}} - \sum_j^N\lambda_{pj}C_{p\xi} = 0,\ee
where
\begin{align*}
\pddt{E[\Psi]}{C^*_{p\xi}} &= \pddt{}{C^*_{p\xi}}\left( \sum_{i\leq N}\sum_{\alpha\beta}C_{i\alpha}^*C_{i\beta}\bra{\alpha}h_0\ket{\beta} +
\frac{1}{2}\sum_{ij\leq N}\sum_{\alpha\beta\gamma\delta}C_{i\alpha}^*C_{j\beta}^*C_{i\gamma}C_{j\delta}\bra{\alpha\beta}v\ket{\gamma\delta}_{AS}\right)\\
&= \sum_{\beta}C_{p\beta}\bra{\xi}h_0\ket{\beta} +
\frac{1}{2}\sum_{ij\leq N}\sum_{\alpha\beta\gamma\delta}\left(\delta_{ip}\delta_{\alpha\xi}C_{j\beta}^*+
C_{i\alpha}^*\delta_{jp}\delta_{\beta\xi}\right)C_{i\gamma}C_{j\delta}\bra{\alpha\beta}v\ket{\gamma\delta}_{AS}\\
&=\sum_{\gamma}C_{p\gamma}\bra{\xi}h_0\ket{\gamma} +
\sum_{j\leq N}\sum_{\beta\gamma\delta}C_{j\beta}^*C_{p\gamma}C_{j\delta}\bra{\xi\beta}v\ket{\gamma\delta}_{AS}\\
&=\sum_\gamma\left(\bra{\xi}h_0\ket{\gamma}+\sum_{j\leq N}\sum_{\beta\delta}C_{j\beta}^*C_{j\delta}\bra{\xi\beta}v\ket{\gamma\delta}_{AS}\right)C_{p\gamma}
\end{align*}
We can define \( \epsilon_p^{hf}=\sum_j\lambda_{pj} \) and
\be\label{hfd}
 h_{\xi\gamma}^{hf}=\bra{\xi}h_0\ket{\gamma}+\sum_{j\leq N}\sum_{\beta\delta}C_{j\beta}^*C_{j\delta}\bra{\xi\beta}v\ket{\gamma\delta}_{AS},
\ee
so that our equations \ref{lag} become
\be\label{hfi}
\sum_\gamma h_{\xi\gamma}^{hf}C_{p\xi} = \epsilon^{hf}_{p}C_{p\xi},
\ee
or, rewriting as a matrix equation
\be\label{hfm}
h^{hf}C = E C,
\ee
where $E$ is the diagonal matrix with $\epsilon^{hf}_p$ as elements.
Now we can see what I meant by the $\Set{\Ket{\psi}}$ states with lowest energy. If we arrange the states as a column vector then
\(\vec{\psi} = C\vec{\phi}\). So, by equation \ref{hfm} \(h^{hf}\vec{\psi} = h^{hf}C\vec{\phi} = EC\vec{\phi} = E\vec{\psi}\), the column vector with elements
\(\epsilon^{hf}_i\ket{\psi_i}\). Now the states that go into the new Slater determinant are the N states with the lowest values for $\epsilon^{hf}$ associated in this way.
The total energy of the new Slater determinant \(\Ket{\Psi}\) is given by equation \ref{Ea} or equivalently
\be\label{Ep}
E[\Psi] = \sum_{i\leq N}\epsilon^{hf}_i -
\frac{1}{2}\sum_{ij\leq N}\sum_{\alpha\beta\gamma\delta}C_{i\alpha}^*C_{j\beta}^*C_{i\gamma}C_{j\delta}\bra{\alpha\beta}v\ket{\gamma\delta}_{AS}
\ee
and is minimized by when $\Set{\Ket{\Psi}}$ is sorted in this way.

\section{Matrix Elements}
In the Hartree-Fock calculations it will be necesary to have the antisymmetrized two-body matrix elements $\bra{\alpha\beta}v\ket{\gamma\delta}_{AS}$.
In general these elements can be calculated before the Hartree-Fock calculations. In the case of a two-dimensional isotropic harmonic oscillator
with Coloumb-interactions there exists an analytical expression for the $\bra{\alpha\beta}v\ket{\gamma\delta}$ found in \cite{analytho}. The expression is rather complicated and I won't
include it here, but we were provided with a c++ function to calculate the elements.

  
\section{Algorithm}
Equation \ref{hfm} is a simple eigenvalue equation, but the matrix  $h^{hf}$ depends on $C$, and is thus not known from the beginning. So we will need a more complicated
method than a simple diagonalisation to solve the problem. One way to solve this is a self consistent field iteration. Here we iterate through
different values for $C$ until we find one that consistently solves equation \ref{hfm}. This is done by the following scheme:
\begin{itemize}
\item
  Choose an initial guess for $C$, $C_0$
\item for n = 0,1,2\ldots
  \begin{itemize}
  \item Find $h^{hf}(C_n)$ by equation \ref{hfd}
  \item Diagonalize $h^{hf}(C_n)$ with $C_{n+1}$ as the eigenvactor matrix
  \item If $C_{n+1}\approx C_n$ break, else continue
  \end{itemize}
\item
  Find $E[\psi]$ from the final $C$ and its eigenvalues by equation \ref{Ep}
\end{itemize}

As mentioned \(\Set{\Ket{\phi}}\) is an infinite set, and so accordingly ought to be infinite-dimensional.
In that case our algorithm consists of infinite operations and takes infinite time to finish.
So, we must choose a subset of \(\Set{\Ket{\phi}}\).
Since we are looking for low energy states it makes sense to choose a the lowest energy states in \(\Set{\Ket{\phi}}\).
Let \(\Set{\Ket{\phi}}_n\) be the set of $\ket{\phi}$ states in the $n$ lowest energy shells. Then there are $n(n+1)$ elements in
\(\Set{\Ket{\phi}}_n\). If we use this subset as our starting basis the matrices have dimension $(n(n+1))^2\sim n^4$. The cost of both constructing
$h^{hf}$ and diagonalizing it ought to scale aproximately as $n^4$, while finding $E[\Psi]$ should scale as $n^8$, thankfully the latter is only done
once.

To find out how many shells to use one can start with the minimal amount of shells, and add one and one shell to the calculations. When the energy seems
to have converged as a function of shells we have reached the so-called Hartree-Fock limit, and found a suitable number of shells.  


\section{Implementation}

I have implemented the above algorithm in a class in the file hartreefock.cpp found at https://github.com/khhelland/Fys4411/tree/master/1 along with
other files relevant to the project. I ran the algorithm to find the energy $E[\Psi]$ for 2,6,12 and 20 particles for up to 10 shells, or
until the Hartree-Fock limit up to $10^{-4}$ with $\omega$ 1 and 0.5. 

\section{Cost}
Figures \ref{figtime1} and \ref{figtime05} show graphs of time used to find $\Ket{\Psi}$ by my program for the different numbers of particles
and frequencies. The cost does not seem fit well with either $n^4$ or $n^8$, which is not very surprising since the algorithm contains different
parts that scale differently as a function of $n$ and an unpredictable iterative element. We see from the figures that the cost for a given
number of shells does not seem to be very sensitive to the number of particles, but as we shall see a higher number of particles requires a
higher number of shells. We see that for nine and ten shells the time used for the calculation is on the order of hours, and it becomes intractable
to work with more shells unless the code is optimized.


\begin{figure}
  \begin{center}
    \includegraphics[width = 12cm]{time1}
  \end{center}
  \caption{Time used to find $\Ket{\Psi}$ by number of shells from \(\Set{\Ket{\Phi}}\) basis used for $\omega=1$.}\label{figtime1}
\end{figure}


\begin{figure}
  \begin{center}
    \includegraphics[width = 12cm]{time05}
  \end{center}
  \caption{Time used to find $\Ket{\Psi}$ by number of shells from \(\Set{\Ket{\Phi}}\) basis used for $\omega=0.5$.}\label{figtime05}
\end{figure}


\section{Benchmarks}


In \cite{mortenslides} there is a table of values for $E[\Psi]$ in the isotropic two dimensional harmonic oscillator with our units and $\omega = 1$
for six particles and different numbers of shells. I have collected the values together with my results for the same calculations in table \ref{tabbench}.
Comparing the two, we see that they agree up to the precision of my results which indicates that my program is functioning correctly.


\begin{table}
  \caption{Table of $E[\Psi]$ from my calculations (``My $E[\Psi]$ [$\omega$]'') and from \cite{mortenslides} ( ``Benchmark $E[\Psi]$ [$\omega$]'') for $\omega = 1$ with 6 particles.}\label{tabbench}
  \begin{center}
    \begin{tabular}{*{3}{c}}
      Shells &  My $E[\Psi]$ [$\omega$] & Benchmark $E[\Phi]$ [$\omega$]   \\
      \hline
      3&21.5932&21.59320\\
      4&20.7669&20.7669\\
      5&20.748&20.7484\\
      6&20.7203&20.72026\\
      7&20.7201&20.71925\\
      8&20.7192&20.71925\\
      9&20.7192&20.71922
    \end{tabular}
  \end{center}
\end{table}



\section{Results}



Figure \ref{figshellsw1} contains graphs of $E[\Psi]$ as a function of number of shells for $\omega = 1$.
For 2 particles the Hartree-Fock limit was reached with 7 shells and for 6 particles with 8 shells.
For 12 and 20 particles my criterion for the Hartree-Fock limit was not reached, but looking at the graphs the energy seems
quite stable from about 8 shells and on.
Figure \ref{figshellsw05} shows the same graphs for $\omega = 0.5$.
For 2 particles $E[\Psi]$ seems to come in pairs so that it is the same for 1 and 2 shells, 3 and 4 shells, and so on.
The energy for 7 and 8 was the same as for 9 shells up to my criterion so I conclude that the Hartree-Fock limit was reached with 7 shells.
The pairing is probably due to some symmetry in the $\bra{\alpha\beta}v\ket{\gamma\delta}$, but I am not sure.
For 6 particles the limit was reached with 8 shells, and for 12 and 20 particles my criterion for the limit was again not reached.

Table \ref{tab} shows the Hartree-Fock limit energies where the limit was reached, or the energies for the final calculations where it was not.



\begin{figure}
  \begin{center}
    \includegraphics[width = 12cm]{shellsw1}
  \end{center}
  \caption{$E[\Psi]$ by number of shells from \(\Set{\Ket{\Phi}}\) basis used for $\omega=1$.}\label{figshellsw1}
\end{figure}

\begin{figure}
  \begin{center}
    \includegraphics[width = 12cm]{shellsw05}
  \end{center}
  \caption{$E[\Psi]$ by number of shells from \(\Set{\Ket{\Phi}}\) basis used for $\omega=0.5$.}\label{figshellsw05}
\end{figure}




\begin{table}
  \caption{Energies for best results.}
  \label{tab}
  \begin{center}
    \begin{tabular}{*{4}{c}}
  Particles & Shells & $\omega$ & $E[\Psi]$ [a.u.] \\
  \hline
  2&7&1&3.16191\\
  2&7&0.5	&2.8386\\
  6&8&1&20.7192\\
  6&8&0.5	&17.8352\\
  12&10&1&66.912\\
  12&10&0.5&56.6229\\
  20&10&1&158.018\\
  20&10&0.5&132.405\\
\end{tabular}

  \end{center}
\end{table}

%% $\left(\begin{matrix}-0.62&0.69&-0.00&-0.00&0.00&0.00&-0.00&0.00&0.00&-0.00&-0.36&0.07\\ 
0.69&0.62&-0.00&-0.00&0.00&0.00&0.00&0.00&-0.00&0.00&0.07&0.36\\ 
-0.00&-0.00&-0.52&-0.72&0.04&-0.45&0.00&0.00&0.00&0.00&-0.00&0.00\\ 
0.00&-0.00&-0.72&0.63&0.25&-0.16&-0.00&0.00&-0.00&-0.00&-0.00&0.00\\ 
-0.00&0.00&0.46&0.17&0.42&-0.76&-0.00&-0.00&0.00&-0.00&0.00&-0.00\\ 
0.00&0.00&-0.00&0.23&-0.87&-0.43&0.00&0.00&-0.00&0.00&-0.00&0.00\\ 
0.00&-0.00&0.00&0.00&0.00&-0.00&-0.00&1.00&0.00&-0.00&-0.00&0.00\\ 
0.00&0.00&-0.00&-0.00&-0.00&-0.00&0.04&-0.00&0.00&-1.00&0.00&-0.00\\ 
0.25&-0.27&0.00&-0.00&0.00&-0.00&0.00&-0.00&-0.00&-0.00&-0.91&0.18\\ 
-0.27&-0.25&0.00&0.00&-0.00&-0.00&-0.00&-0.00&0.00&-0.00&0.18&0.91\\ 
-0.00&-0.00&0.00&0.00&0.00&0.00&1.00&0.00&0.00&0.04&0.00&0.00\\ 
0.00&0.00&-0.00&0.00&-0.00&-0.00&-0.00&-0.00&1.00&-0.00&-0.00&-0.00\\ 
\end{matrix}\right)$


\section{Discussion}
Allthough the Hartree-Fock limit was not reached in all cases the graphs seem to be close to converging for all numbers of particles
and both frequencies. Thus it seems that we have found states that are close to ideal for this method. However, it is not at all clear how close these
states are to the true ground states, it may well be that only looking at the space of slater determinants of superpositions of non-interacting states
limits the search space too much, and that the true ground state is very different from anything in our search space.

Comparing values in table \ref{tab} clearly the energies are lower for $\omega = 0.5$ than for $\omega = 1$. Much of this difference is explained by
the difference in $\Braket{H_0}$ where according to equation \ref{eqH0} the \(\Braket{H_0}_{\omega = 1} = 2\Braket{H_0}_{\omega = 0.5}\).
I have collected the interaction energy, by which I do not mean $\Braket{H_I}$ for any state, but rather the difference between
$\Bra{\Phi}H_0\Ket{\Phi}$ and $\Bra{\Psi}H\Ket{\Psi}$, in table \ref{tabI}. There we see that most of the difference in $E[\Psi]$ for different $\omega$
is due to the difference
in $\Braket{H_0}$. 
Still there is a noticable difference in the interaction energy and it is always bigger for $\omega = 0.5$ than it is for
$\omega = 1$, although I do not see any trend in how big this difference is, and it remains small for many particles.

As for the difference in $E[\Psi]$ for different number of particles we see by comparing tables \ref{tab} and \ref{tabI}
that the difference due to interaction energy grows much faster than the difference in the non-interacting case.
Allthough there are only 2 times 4 data-points table \ref{tabI} seems to fit with the interaction energy depending on
the square of the number of particles. This makes sense, as looking at equation \ref{eqHi} it ought to depend on the number of pairs of particles
and this goes as the square. At the same time looking at equation \ref{eqH0} we see that $\braket{H_0}$ goes as number of filled shells cubed,
while number of particles goes as number of filled shells squared. So for $n$ paricles $\braket{H_0}$ ought to go as $n^{\f{3}{2}}$, at least when $n$
becomes large. Thus for large $n$ the interaction energy should dominate.


\begin{table}
  \caption{Interaction energy: That is the difference between $\Bra{\Phi}H_0\Ket{\Phi}$ and $\Bra{\Psi}H\Ket{\Psi}$ for
    different numbers of particles. $E_I(\omega)$ is $\Bra{\Psi}H\Ket{\Psi}-\Bra{\Phi}H_0\Ket{\Phi}$ and
    \(\Delta E_I\) is \(|E_I(1)-E_I(0.5)|\)}\label{tabI}
  \begin{center}
    \begin{tabular}{*{4}{c}}
      Particles &  $E_I(1)$ & $E_I(0.5)$ & $\Delta E_I$ \\
      \hline
      2&1.162 & 1.839 & 0.677\\
      6&10.719&12.835 &2.12\\
      12&38.912 & 42.623 & 3.71\\
      20&98.02 & 102.41 & 2.41\\
    \end{tabular}
  \end{center}
\end{table}




\section{Conclusion}
I have found aproximations to the ground state of $H$ for 2,6,12 and 20 particles and $\omega = 1$ and $0.5$. For 2 and 6 particles I reached the
Hartree-Fock limit. That means that my aproximation is the ideal (up to my criterion for the limit) ground state based on a Slater determinant
of linear combinations of $\ket{\phi}$ states. For 12 and 20 particles my criterion was not reached, but looking at the figures \ref{figshellsw1}
and \ref{figshellsw05} it seems that it was not that far from being reached. So my aproximation should be close to ideal with this method for these
cases as well. However it may well be that the best slater determinant is not a very good aproximation, so the fact that my aproximations are close
to the best Slater determinants does not necessarily mean that they are good aproximations to the true ground state.

\begin{thebibliography}{1}


\bibitem{analytho}
  E. Anisimovas and A. Matulis, ``Energy spectra of few-electron quantum dots.'', J. Phys.,10:601 615, 1998.
\bibitem{mortenbok}
  M. Hjort-Jensen, \emph{Computational Physics: Lecture Notes Fall 2015}, found at
  https://github.com/CompPhysics/ComputationalPhysics2/blob/gh-pages/doc/Literature/lectures2015.pdf
  on the 31/01/17
\bibitem{mortenslides}
  M. Hjort-Jensen,\emph{Computational Physics: Hartree-Fock methods and introduction to Many-body Theory}
  found at http://compphysics.github.io/ComputationalPhysics2/doc/pub/basicMB/pdf/basicMB-print.pdf
  on the 16/03/17
\bibitem{simen}
  Simen Kvaal, \emph{Lecture notes for Fys-Kjm4480: Quantum Mechanics for many-particle systems},
  october 2016, not published.
\end{thebibliography}



%% \begin{table}
%%   \caption{}\label{kal1}
%%   \begin{center}
%%     \begin{tabular}{|r|r|}
      
%%       \hline
%%       \textbf{masse/g}& \textbf{Utslag/mm}\\ \hline
%%     \end{tabular}
%%   \end{center}
%% \end{table}
      




%% \begin{figure}
%%   \begin{center}
%%     \fbox{\includegraphics[width = 12cm]{}}
%%   \end{center}
%%   \caption{}\label{}
%% \end{figure}


\end{document}
