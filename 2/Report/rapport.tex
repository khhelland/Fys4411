\documentclass[a4paper,norsk,10pt]{article}
\usepackage[utf8]{inputenc}
\usepackage[T1]{fontenc}
\usepackage[norsk]{babel}
\usepackage{graphicx,mathpple, textcomp, varioref}
\usepackage{fullpage}
\usepackage{fancyhdr}
\usepackage{lastpage}
\usepackage{hyperref}
\usepackage{amsmath}
\usepackage{braket}
\usepackage{enumitem}

\title{Fys4110: Project 2}
\author{Knut Halvor Helland}
\pagestyle{fancyplain}
\fancyhf{}
\renewcommand{\headrulewidth}{0pt}
\fancyfoot[R]{\thepage/\pageref{LastPage}}
\tolerance = 5000
\hbadness = \tolerance
\pretolerance = 2000
\setlength{\headheight}{20pt}

\newcommand{\unit}[1]{\; \mathrm{#1}}
\newcommand{\bb}[1]{\boldsymbol{#1}}
\newcommand{\p}{\partial}
\newcommand{\dd}{\mathrm{d}}
\newcommand{\ddt}[2]{\frac{\dd #1}{\dd #2}}
\newcommand{\dndt}[3]{\frac{\dd^{#3} #1}{\dd #2^{#3}}}
\newcommand{\pddt}[2]{\frac{\p #1}{\p #2}}
\newcommand{\pndt}[3]{\frac{\p^{#3} #1}{\p #2^{#3}}}
\newcommand{\Rar}{\Rightarrow}
\newcommand{\rar}{\rightarrow}
\newcommand{\lagr}{\mathscr{L}}
\newcommand{\ham}{\mathcal{H}}
\newcommand{\id}{\bb{1}}
\newcommand{\deldt}[2]{\frac{\delta #1}{\delta #2}}
\newcommand{\be}{\begin{equation}}
\newcommand{\ee}{\end{equation}}
\newcommand{\f}{\frac}


\renewcommand{\bar}{\overline}


\begin{document}
\maketitle{}


\appendix
\section{Derivatives of the 2 particle trial wavefunction}
In the course of the project we needed analytical expressions for different derivatives of the 2 particle trial wavefunction. I have collected
the differentiations here.
Trial wave function is
\[\psi_T(\bb{r_1},\bb{r_2}) = C\exp\left(-\alpha\omega(r_1^2+r_2^2)/2\right)\exp\left(\f{ar_{12}}{1+\beta r_{12}}\right).\]

\subsection{Gradient}
In order to compute the driftforce for importance sampling we needed
\[\f{1}{\psi_T}\ddt{\psi_T}{z},\]
where \(z_i = x_1,x_2,y_1,y_2\).
As \(\psi_T\) is an exponential
\begin{align*}
  \f{1}{\psi_T}\ddt{\psi_T}{z_i} &= \ddt{}{z_i}\left(-\f{1}{2}\alpha\omega(r_1^2+r_2^2) + \f{ar_{12}}{1+\beta r_{12}}\right)\\
  &= -\alpha\omega z_i + a\left(\f{1}{1+\beta r_{12}} - \f{r_{12}\beta}{(1+\beta r_{12})^2}\right)\pddt{r_{12}}{z_i}\\
  &= -\alpha\omega z_i + \f{a}{(1+\beta r_{12})^2}\pddt{r_{12}}{z_i}\\
  &= -\alpha\omega z_i + \f{a}{(1+\beta r_{12})^2}\f{(z_i-z_j)}{r_{12}},\\
\end{align*}
where \(z_j\) is the coordinate of the other particle along the same dimension as \(z_i\), so for example when
\(z_i=x_2\), \(z_j = x_1\).

\subsection{Laplacian}

The local energy is defined as
\[E_L(\bb{r_1},\bb{r_2}) = \f{1}{\psi_T}H\psi_T.\]
In our case the Hamiltonean is
\[H = \sum_{i=1}^2-\f{1}{2}\nabla_i^2+\f{1}{2}\omega^2r_i^2 + \f{1}{r_{12}}.\]
The laplacian is
\[\sum_i\nabla^2 = \sum_i\pndt{}{z_i}{2},\]
where \(z_i\) is the same as above.

Again since \(\psi_T\) is an exponential we have
\begin{align*}
  \f{1}{\psi_T}\pndt{\psi_T}{z_i}{2} &= \pndt{}{z_i}{2}\left(-\alpha\omega(r_1^2+r_2^2)/2 + \f{ar_{12}}{1+\beta r_{12}}\right) +
    \left(\pddt{}{z_i}\left(-\alpha\omega(r_1^2+r_2^2)/2 + \f{ar_{12}}{1+\beta r_{12}}\right)\right)^2\\
    &= \pddt{}{z_i}\left(-\alpha\omega z_i + \f{a}{(1+\beta r_{12})^2}\f{(z_i-z_j)}{r_{12}}\right) +
    \left(-\alpha\omega z_i + \f{a}{(1+\beta r_{12})^2}\f{(z_i-z_j)}{r_{12}}\right)^2\\
    &= -\alpha\omega + \f{a}{(1+\beta r_{12})^2r_{12}} - \f{2a\beta(z_i-z_j)^2}{(1+\beta r_{12})^3r_{12}^2}
    - \f{a(z_i-z_j)^2}{(1+\beta r_{12})^2r_{12}^3} + \alpha^2\omega^2z_i^2 -
    \f{2\alpha\omega a z_i(z_i-z_j)}{(1+\beta r_{12})^2r_{12}} + \f{a^2(z_i-z_j)^2}{(1+\beta r_{12})^4r_{12}^2}.
\end{align*}
So, because the $\psi_T$ is symmetric under $1\leftrightarrow 2$ and $x\leftrightarrow y$:

\[\f{1}{\psi_T}\sum_i\nabla^2_i\psi_T = \alpha\omega(r_1^2 + r_2^2) -4\alpha\omega  + \f{2a^2}{(1+\beta r_{12})^4}+  \f{4a}{(1+\beta r_{12})^2r_{12}} - \f{4a\beta}{(1+\beta r_{12})^3} -
\f{2a}{(1+\beta r_{12})^2r_{12}} -\f{2\alpha\omega a r_{12}}{(1+\beta r_{12})^2}\]
and finally:
\[\f{1}{\psi_T}\sum_i\nabla^2_i\psi_T = \alpha\omega(r_1^2 + r_2^2) -4\alpha\omega  + \f{2a}{(1+\beta r_{12})^2}\left[\f{a}{(1+\beta r_{12})^2}+  \f{1}{r_{12}} - \f{2\beta}{(1+\beta r_{12})} -\alpha\omega r_{12}\right]\]

Using this expression we have the local energy as
\[E_L(\bb{r_1},\bb{r_2}) = \left[\f{1}{2}(1- \alpha)\omega(r_1^2 + r_2^2) + 2\alpha\omega\right]  - \f{a}{(1+\beta r_{12})^2}\left[\f{a}{(1+\beta r_{12})^2}+  \f{1}{r_{12}} - \f{2\beta}{(1+\beta r_{12})} -\alpha\omega r_{12}\right] + \f{1}{r_{12}},\]
where the second term contains the  terms from the  Jastrow-factor and the cross term and the third is the interaction term.

\subsection{Derivatives w.r.t \(\alpha\) and \(\beta\)}
In order to find the optimal parameters \(\alpha\) and \(\beta\) we needed the derivatives of \(\psi_T\) with respect to these.
\[  \f{1}{\psi_T}\pddt{\psi_T}{\alpha} = -\f{1}{2}\omega(r_1^2 + r_2^2).\]
\[  \f{1}{\psi_T}\pddt{\psi_T}{\beta} = -\f{ar_{12}^2}{(1+\beta r_{12})^2}.\]

  
%% \begin{table}
%%   \caption{}\label{kal1}
%%   \begin{center}
%%     \begin{tabular}{|r|r|}
      
%%       \hline
%%       \textbf{masse/g}& \textbf{Utslag/mm}\\ \hline
%%     \end{tabular}
%%   \end{center}
%% \end{table}
      




%% \begin{figure}
%%   \begin{center}
%%     \fbox{\includegraphics[width = 12cm]{}}
%%   \end{center}
%%   \caption{}\label{}
%% \end{figure}


\end{document}
